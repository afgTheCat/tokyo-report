\documentclass{article}
\usepackage{arxiv}
\usepackage[utf8]{inputenc} % allow utf-8 input
\usepackage[T1]{fontenc}    % use 8-bit T1 fonts
\usepackage{hyperref}       % hyperlinks
\usepackage{url}            % simple URL typesetting
\usepackage{booktabs}       % professional-quality tables
\usepackage{amsfonts}       % blackboard math symbols
\usepackage{nicefrac}       % compact symbols for 1/2, etc.
\usepackage{microtype}      % microtypography
\usepackage{lipsum}
\usepackage{amsthm}
\usepackage{amsmath}
\usepackage{minted}
\usepackage{graphicx}
\usepackage{caption}
\usepackage[thinc]{esdiff}

\newtheorem{theorem}{Theorem}
\newtheorem{lem}[theorem]{Lemma}
\newtheorem{proposition}[theorem]{Proposition}
\theoremstyle{definition}
\newtheorem{definition}{Definition}
\newtheorem{condition}{Condition}

\title{Research project}

\examineenumber{11070}
\author{Gábor Ádám Fehér}

\begin{document}
\maketitle

\newpage

\section*{Answer to Theme A}
From the fundamental concepts related to mathematical informatics we present the \textbf{Cooley–Tukey algorithm}. The algorithm allows us to compute the discrete Fourier transform of a \(n\) long sequence in \(\mathcal{O}(n\log{}n)\) time. Algorithm with such properties are called fast Fourier transform (FFT) algorithms. Among the many variants of the Cooley–Tukey algorithm, the \textbf{radix-2 DIT} is the simplest and most common one, and thus we present it in great detail. Other forms of the algorithm, as well as other types of FFT algorithms, are also mentioned but no rigorous construction is given.

\subsection*{Mathematical overview}
\begin{definition}[Discrete Fourier transform]
    The discrete Fourier transform (DFT) over the \(n\) dimensional complex field is an invertible linear transformation \(\mathcal{F}: \mathbb{C}^n \to \mathbb{C}^n\). The output of the function for \({(x_k)}_{0 \leq k \leq n - 1}\) is defined by
    \begin{gather}
        X_k = \left(\mathcal{F}(x)\right)_k = \sum_{j=0}^{n-1} x_j e^{-\frac{2{\pi}ikj}{n}} =  \sum_{j=0}^{n-1} x_j \left[ \cos\left( \frac{2{\pi}ikj}{n} \right) - i \sin\left( \frac{2{\pi}ikj}{n} \right) \right],
    \end{gather}
    where the right side of the equation is due to Euler's formula and the trigonometric properties of sine and cosine.
\end{definition}

We should mention that the sign of the exponential is sometimes taken as positive. The resulting equation is equal to the inverse of the previously defined Fourier transformation multiplied by the constant \(n\), that is
\begin{gather}
    x_k = \left(\mathcal{F}^{-1}(X)\right)_k = \frac{1}{n}\sum_{j=0}^{n-1}X_j e^{\frac{2{\pi}ikj}{n}}.
\end{gather}
With some minor adjustments the Cooley–Tukey algorithm is capable of computing the inverse-DFT of the \(n\) long sequence, so no matter which convention we use, the algorithm remains useful.

We introduce some notations. Given \(x = (x_k)_{0 \leq 2n-1}\), we define \(X^{[0]}\) to be the DFT of the even-indexed terms, while \(X^{[1]}\) to be the DFT of the odd-indexed terms in the sequence. That is
\begin{gather}
    X^{[0]}_k = \left(X^{[0]}\right)_k = \left(\mathcal{F}(x_0, x_2, \dots x_{2n-2})\right)_k = \sum_{j=0}^{n-1} x_{(2j)} e^{-\frac{2{\pi}ikj}{n}} = \sum_{j=0}^{n-1} x_{(2j)} e^{-\frac{4{\pi}ikj}{2n}} \label{dq:1} \\
    X^{[1]}_k = \left(X^{[1]}\right)_k = \left(\mathcal{F}(x_1, x_3, \dots x_{2n-1})\right)_k = \sum_{j=0}^{n-1} x_{(2j+1)} e^{-\frac{2{\pi}ikj}{n}} = \sum_{j=0}^{n-1} x_{(2j+1)} e^{-\frac{4{\pi}ikj}{2n}}. \label{dq:2}
\end{gather}
The key observation to make in order to verify the validity of the algorithm is
\begin{gather}
    X_k = \sum_{j=0}^{2n-1} x_j e^{-\frac{2{\pi}ikj}{2n}} = \sum_{j=0}^{n-1} x_{(2j)} e^{-\frac{4{\pi}ikj}{2n}} + e^{-\frac{2{\pi}ik}{2n}}\sum_{j=0}^{n-1} x_{(2j+1)} e^{-\frac{4{\pi}ikj}{2n}}. \label{dq:4}
\end{gather}
Thus for \(0 \leq l \leq n-1\) we have
\begin{gather}
    X_l = X^{[0]}_l + e^{-\frac{2{\pi}ik}{2n}} X^{[1]}_l, \text{ and }
    X_{n+l} = X^{[0]}_l - e^{-\frac{2{\pi}ik}{2n}} X^{[1]}_l, \label{dq:3}
\end{gather}
since
\begin{gather}
    \sum_{j=0}^{n-1} x_{(2j)} e^{-\frac{4{\pi}i(n+l)j}{2n}} = \sum_{j=0}^{n-1} x_{(2j)} e^{-\frac{4{\pi}ilj}{2n}}, \text{ and } e^{-\frac{2{\pi}i(n+l)}{2n}} = -e^{-\frac{2{\pi}il}{2n}}.
\end{gather}

\subsection*{The algorithm}
This basis of all variants of the Cooley–Tukey algorithm is the divide-and-conquer technique. In the radix-2 case the size of the input has to be a power of two. We may add padding zeros to fit the size constraint. From this point on we assume that input vectors are of the right size. Given an \({x}_{0\leq k \leq n}\), where \(2\leq n\), equations \ref{dq:1} and \ref{dq:2} indicate that the input vector should be divided into two parts: the even-indexed and the odd-indexed terms. Once the DFTs of the split vector are calculated, via recursive calls, the DFT of the input vector can be calculated by using equation \ref{dq:3}. If the input vector is one dimensional, its DFT is itself. The following Python code is a working implementation of the radix-2 case, illustrating the key ideas used, but due to its performance it is not meant to be used in real world applications.
\inputminted[xleftmargin=20pt, linenos]{python}{./code/recursive_fft.py}
To evaluate the time complexity of the algorithm, apart from the recursive calls in line 13 and 14, the algorithm runs in \(\Theta(n)\) time, where \(n\) is the length of the input. Using the recurrence relation
\begin{gather}
    T(n) = 2T(n/2) + \Theta(n) = \Theta(n\log{}n).
\end{gather}
With little modification, the Cooley–Tukey algorithm can compute the inverse-DFT of a transformed vector. Similarly to equation \ref{dq:4}, we can derive
\begin{gather}
    x_k = \sum_{j=0}^{2n-1} X_j e^{\frac{2{\pi}ikj}{2n}} = \sum_{j=0}^{n-1} X_{(2j)} e^{\frac{4{\pi}ikj}{2n}} + e^{\frac{2{\pi}ik}{2n}}\sum_{j=0}^{n-1} X_{(2j+1)} e^{\frac{4{\pi}ikj}{2n}}
\end{gather}
so in the 9th line, instead of \(e^{-2{\pi}i/N}\) we have \(e^{2{\pi}i/N}\) for the \(N\) long input, and by multiplying each term of the end result by \(1/n\) given the initial input had size of \(n\), we obtain the original \(x\), within the same time complexity.
\subsection*{Importance and Usage}
The FFT is widely used among different fields of engineering, computer science and mathematics. 

Many popular compression methods, such as \texttt{jpeg} or \texttt{webm}, use discrete cosine transforms (DCT). Algorithms that compute DCTs in \(\mathcal{O}(n\log{}n)\) are known as fast cosine transform (FCT) algorithms. Specialized FCT algorithms exist, that directly compute a vectors DCT, and tough the theoretical running time of such algorithms are better, on modern hardware computing DCTs via FFTs with some \(\mathcal{O}(n)\) pre- and post-processing steps are often faster.

When two degree-bound \(n\) polynomials are represented in the a "extended" point-value form, over \(2n\) distinct points, multiplication can be done in \(\mathcal{O}(n)\) time, by simply multiplying the coefficients. Representing such a polynomial over the \(n\)th root of unity is the same as having the coefficient vectors transformed. Thus multiplying two polynomials can be done in \(\mathcal{O}(n\log{}n)\) time by using FFTs: first transforming the polynomials to their "extended" point value form in \(\mathcal{O}(n\log{}n)\) time, multiplying them in \(\mathcal{O}(n)\), and then transforming back the result to coefficient form. Polynomial representation of the Toeplitz matrices also allow us to multiply them within the same time complexity.

FFts are also widely used in theoretical results. For a long time the \textbf{Schönhage–Strassen algorithm} was the fastest known algorithm for multiplying two large integers. With running time \(\mathcal{O}(n\log{}n\log{}\log{}n)\), it is asymptotically faster than other many traditional multiplication methods, like for example the \textbf{Karatsuba algorithm}, but it only starts to outperform them for numbers with 10000 to 40000 digits, thus it is rarely used in practice. The algorithm employs modular arithmetic instead of the roots of unity (Fourier transforms can be performed in any algebric ring). In 2019 Harvey and van der Hoeven published a \(\mathcal{O}(n\log{}n)\) algorithm, which also builds upon on FFTs. However this is a galactic algorithm, so it only bears theoretical importance.

\section*{Answer to Theme B}
When studying stochastic processes, the Markov property is a commonly made assumption. As many real life processes have memory, the Markov property can be a limiting. Excited random walks (ERWs) are a class of non-Markovian stochastic processes that generalize other well understood processes. As a novel field of mathematics, results described here are theoretical in nature, meanwhile other non-Markovian processes are already widely used, such as the ARMA model in finance and the fractional Brownian motion in biology and climatology. Some established results can be viewed from the perspective of ERWs. Here we are concerned with ERWs on the \(d\) dimensional integer lattice but the construction of ERWs can be extended to other graphs or continuous time-space processes. As rigorous introduction to the subject is tedious, we only describe the intuitive meaning behind the ERWs.

\subsection*{Excited random walks}
ERWs can be thought of as a generalization of simple random walks (biased or symmetric) or random walks in random environments. On each site of the \(d\) dimensional integer lattice an infinite stack of \textit{cookies} is placed. A cookie is a \(2d\) dimensional vector that describes a probability distribution on the nearest neighbor transitions. The walker performs a random walk on the integer lattice. Upon hitting a position it consumes the cookie on top and moves according to the probability distribution the cookie describes. We call a specific configuration of cookies a \textit{cookie-environment}. The set of all cookie-environments is denoted by \(\Omega\), while a specific cookie environment is usually denoted by \(\omega\). As per definition, \(\omega \in \Omega\). The set of \(d\)-dimensional unit coordinate vectors is denoted by \(\mathcal{M}\) and the probability of moving towards the vector \(e \in \mathcal{M}\) at position \(z\) upon the \(n\)th arrival is denoted by \(\omega(z, e, n)\). That is the movement of the ERW \(X\) starting in position \(x\) is governed by the following equations:
\begin{gather}
    P_{x,\omega}[X_0 = x] = 1 \\
    P_{x, \omega}[X_{n+1} = X_n + e \mid (X_i)_{0 \leq i \leq n }] = \omega(X_n, e, \# \{ i \in \{0, 1, \dotso, n\} \mid X_i = X_n \} ).
\end{gather}
We call the above \(P_{x,\omega}\) the \textit{quenched} measure. It is easy to see how certain cookie-environments can be equated to the previously given random walks by putting cookies of the same kind on top of each other at each site. We may assume some probability distribution \(\mathbb{P}\) on the cookie-environments itself, further generalizing the model. If we want to investigate a specific environment \(\omega \in \Omega\), we can condition that \(\mathbb{P}[\omega] = 1\). Measures on the cookie-environments enables us to create convenient scenarios like the cookie stacks being independent of each other. We also define the \textit{averaged} measure \(P_x[\cdot]\) as \(\mathbb{E}[P_{x,w}[\cdot]]\).

\subsection*{Restrictions on ERWs}
As ERWs describe a very broad setting for random walks, we often make certain assumptions on \(\mathbb{P}\). The first two conditions are concerned with the relationship between individual cookie stacks. After the more obvious i.i.d. condition we give a weaker one based on stationarity and ergodicity. In this context ergodicity means that \(\forall \mathcal{A} \in \mathcal{F}\) that are invariant under shifts on \(\mathbb{Z}^d\) satisfy \(\mathbb{P}[A] \in \{0, 1\}\), where \(\mathcal{F}\) is the sigma algebra generated by \(\Omega\).
\begin{condition}[IID]\label{IID}
    This condition holds if the family \((\omega(z, \cdot, \cdot))_{z \in \mathbb{Z}^d}\) of cookie stacks is i.i.d. under \(\mathbb{P}\).
\end{condition}
\begin{condition}[SE]\label{SE}
    This condition holds if the family \((\omega(z, \cdot, \cdot))_{z \in \mathbb{Z}^d}\) of cookie stacks is stationary and ergodic with respect to shifts on \(\mathbb{Z}^d\) under \(\mathbb{P}\). 
\end{condition}
We describe a series of elliptic conditions. They are often used to escape degenerate situations. We describe first the weakest of such, and move towards stronger elliptic conditions.
\begin{condition}[WEL]\label{WEL}
    This condition holds if for any \(z \in \mathbb{Z}^d, e \in \mathcal{M}\) there exists an \(i \in \mathbb{N}\) such that \(\omega(z, e, i) > 0\) holds almost surely under \(\mathbb{P}\).
\end{condition}
\begin{condition}[EL]\label{EL}
    This condition holds if for any \(z \in \mathbb{Z}^d, e \in \mathcal{M}\) and \(i  \in \mathbb{N}\) the equation \(\omega(z, e, i) > 0\) holds almost surely under \(\mathbb{P}\). 
\end{condition}
\begin{condition}[UEL]\label{UEL}
    This condition holds if there is a \(\kappa > 0\) such that for all \(z \in \mathbb{Z}^d, e \in \mathcal{M}\) and \(i \in \mathbb{N}\) the equation \(\omega(z, e, i) \leq \kappa\) holds almost surely under \(\mathbb{P}\).
\end{condition}
According to our knowledge the last condition has not been described in the literature. It is inspired by the work done in section 2 of , and is useful when dealing with the range of ERWs.
\begin{condition}[BC\(_F\)]\label{BC}
    The condition holds for \(F \subseteq \mathcal{M}\) if \(\mathbb{P}\left[ \forall e \in F : \sum_{i \geq 1} \omega(0, e, i) < \infty \right] = 0\).
\end{condition}
\subsection*{Results concerning the general case and properties of ERWs}
Some of elementary properties of the ERWs have been proven under relatively weak assumptions on \(\mathbb{P}\). Results of this section were first given in section 2 and 3 of \cite{KZ13}. We start with the results connected to the range of the ERWs.
\begin{theorem}[The range of one dimensional ERWs]\label{thm:erwrange}
    Assume conditions \ref{SE} and \ref{EL} hold. Assume that \ref{BC} also holds for either \(F = \{-1\}\) or for \(F = \{1\}\) but not for both. Then \(\lim_{n\to\infty} X_n \to e \cdot \infty\) holds \(P_0\)-almost surely, where \(\{e\} = F\). If condition \ref{BC} holds for \(F = \{-1, 1\}\), then the range of the random walk is \(P_0\)-almost surely infinite. If condition \ref{BC} does not hold for any \(F \subseteq \mathcal{M}\) then the range of the random walk is \(P_0\)-almost surely finite.
\end{theorem}
A similar theorem can be formulated for the multidimensional case by strengthening condition \ref{SE} to \ref{IID}. Whether the following theorem holds under condition \ref{SE} is to the best of our understanding is an open question.
\begin{theorem}[The range of the multidimensional ERWs]
    Assume conditions \ref{IID} and \ref{EL} hold. If there is an orthogonal set \(F \subset \mathcal{M}\) such that condition \ref{BC} holds for this set, then range of the ERW is \(P_0\)-almost surely infinite. On the contrary, if no such \(F\) exists, than its range is \(P_0\)-almost surely finite.
\end{theorem}
The next theorem is a result regarding transience and recurrence. ERWs are not Markov chains, so the standard definition of recurrence and transience do not apply here. We start by stating what transience and recurrence mean in this setting.
\begin{definition}
    The ERW \(X\) is \textit{recurrent} if \(P_0\)-almost surely every site in \(\mathbb{Z}^d\) is visited infinitely often. It is \textit{transient} if \(P_0\)-almost surely no site is visited infinitely often or equivalently if \(P_0\)-almost surely \(\lim_{n\to\infty} |X_n| \to \infty \).
\end{definition}
Transience in a direction can also be defined analogously by having \(X_n \cdot \ell \to \infty\) for some \(\ell \in \mathbb{Z}^d\) as \(n \to \infty\). The following theorem is a 0-1 law for transience and recurrence in one dimension.
\begin{theorem}[Transient and recurrence of the one dimensional ERWs]
    Assume conditions \ref{SE} and \ref{EL} hold. Then the ERW is \(P_0\)-almost surely recurrent or transient or has finite range.
\end{theorem}
In terms of ERWs we say that \(X\) satisfies a law of large numbers if \(\lim_{n\to\infty} X_n/n = v\) exists. This limit is called the speed of the ERW. The ERW is said to be ballistic if it satisfies the law of large numbers with non zero speed. In the subsequent sections we turn our attention to some of the more specific models of ERWs.
\subsection*{The original ERW}
The original ERW model was introduced by Benjamini and Wilson in \cite{BW03}. The model is constructed on \(\mathbb{Z}^{d}\). The walker gets biased at the first arrival to a site in a direction \(e \in \mathcal{M}\), while on subsequent visits it jumps to a uniformly chosen neighbor. This model can be generalized by fixing some \(\ell \in \mathbb{R}^d\) and assume that cookies induce a drift in that direction. In \cite{MPRV}, a model satisfying conditions \ref{IID} and \ref{UEL} is given, such that
\begin{gather}
    \exists \lambda \in \mathbb{R}^+: \sum_{e\in\mathcal{M}}\omega(0, e, 1)e\cdot \ell \geq \lambda \quad \mathbb{P}\text{-a.s. and } \\
    \omega(0, e, i) = \omega(0, -e, i) \quad \mathbb{P}\text{-a.s. for all } e \in \mathcal{E}, i \leq 2.
\end{gather}
Given that conditions \ref{IID} and \ref{UEL} hold, in \cite{BR07} the model has shown to be ballistic and its speed satisfies \(v \cdot l > 0\).
\subsection*{Any number of \(\ell\)-positive cookies per site}
This model further generalizes the previous one by lifting the restriction on the number of non-placebo cookies, still requiring every cookie to induce a drift in the same \(\ell \in \mathbb{Z}^d\setminus\{0\}\) direction. In the one-dimensional case, we usually assumption \ref{SE} usually hold, while in the multi-dimensional case assumptions \ref{IID} and \ref{UEL} are usually hold. Then
\begin{gather}
    \sum_{e\in\mathcal{M}}\omega(0, e, i)e \cdot \ell \geq 0 \quad \mathbb{P}\text{-a.s. for all } i \in \mathbb{N}.
\end{gather}
If we assume that condition \ref{SE} holds, the one dimensional version of the model has also shown to obey the strong law of large numbers. 

\subsection*{Boundedly many positive or negative cookies per site}
This is probably the most widely studied model of ERWs. It was introduced in \cite{BS08}. Assume conditions \ref{IID} and \ref{WEL} hold and that there is fixed bound after which there are only placebo cookies left in the each cookie stack. In one dimension, using a connection with branching processes, it has been shown that the process obeys the strong law of large numbers. The exact circumstances under which the model is ballistic are also known. Limiting theorems in connection to Brownian motion have also been devised.

\newpage

\section*{Research plan}
My main field of interest is probability theory. As the theoretical foundation for reinforcement learning is the Markov decision process process so I have also spent considerable time learning about the subject. As I would not consider it to be a subfield of probability theory, I will write about those separately.
\subsection*{Probability theory}
My main research interest in probability theory are the stochastic processes. I find non-Markovian processes such as excited random walks or reinforced random walks particularly intriguing. From what I understand, neuron firings can be modeled with reinforced random walks. As such I would be very happy if I could use my knowledge of stochastic processes in the pursuit of understanding the human brain. In \textbf{Theme C} I was tasked to describe how mathematical informatics can help preparing for infectious diseases. Tough previously I did not know anything about the subject, I really enjoyed learning about it. The sources I looked into mostly used differential equations when modeling the spread of viruses/bacterias. I believe that stochastic models can be just as useful when modeling such events as using more traditional approaches and would like to continue learning about these techniques (I even created a short simulation based on competing exponential variables, their memorylessness and the distribution of their minimum value to turn the problem into a five dimensional stochastic progress).

Another branch of probability theory I would like to further learn about is random graphs. Understanding them in my opinion is very useful as they can model interconnected systems such as the internet, forest fires, societies etc., and thus allows us to tackle related problems such as internet routing, fire suppression or prevent the spread of deadly diseases (In my ad hoc simulation I used the Watts–Strogatz model).
\newpage

\section*{Answer to Theme C}
Mathematical informatics can help immensely to prepare for and to handle infectious diseases. Though the tools of bioinformatics are useful to investigate properties of the pathogens and to develop vaccines against it, in this section we choose to focus on other methods to combat epidemics. According to Ming et al, epidemic containment efforts have four phases: \textbf{preparedness}, \textbf{outbreak investigation}, \textbf{response} and \textbf{evaluation}. These phases contain a wide range of tasks, such as inventory management for essential medical supply, surveillance system design, the scheduling of vehicles used to transport medical supply and identification of potential bottlenecks.

\subsection*{The SEIQRS model}
Compartment models are widely used in modeling the spread of infectious diseases. In case of an outbreak, each member of the population is assigned with a label depending on his/her relation with the infectious agent. Individuals with the same label form a compartment. As the epidemic progresses, individuals can also progress between compartments. During the recent COVID-19 outbreak people were divided into 5 groups: susceptible people \((S)\), people during incubation period \((E)\), infectious people \((I)\), quarantined people \((Q)\) and recovered people \((R)\). The name of this model is the \textbf{SEIQRS model}. The underlying social structure may taken to be homogenous, but we use small-world networks in our model. Even tough stochastic frameworks have been introduced, we use the more common deterministic approach of utilizing differential equations to mimic the movement between different compartments. The following figure shows the arrangement of the compartments and how people can move between them.
\begin{figure}[H]
    \centering
    \def\svgwidth{0.7\columnwidth}
    \input{./figures/model_seiqr.pdf_tex}
    \captionsetup{justification=centering,margin=2cm}
    \caption{The SEIQRS model}
    \label{fig:seiqrs}
\end{figure}
The number of susceptible people at time \(t\) is \(S(t)\), and the number of people in each compartments is denoted analogously. On the above figure, \(\langle k \rangle\) is the average degree distribution of the underlying small-world network, \(\hat{S}\) and \(\hat{I}\) abbreviate \(S(t-\tau)\) and \(I(t-\tau)\) respectively, where \(\tau\) is the incubation period. The speed with which susceptible people progress to other compartments is
\begin{gather}
    \diff{S}{t} = -\beta\langle k \rangle S(t)I(t) + \gamma R(t).
\end{gather}
Looking at figure \ref{fig:seiqrs} the other differential equations can be derived easily. As obtaining the analytical solution for the compartment model remains difficult, we have to rely on computational simulations to investigate further properties of the model. The following figure was created using reasonable input parameters with \texttt{jitcdde}, \texttt{numpy} and \texttt{matplotlib}.
\begin{figure}[H]
    \centering
    \def\svgwidth{0.5\columnwidth}
    \input{./figures/all_compartments.pdf_tex}
    \captionsetup{justification=centering,margin=2cm}
    \caption{The SEIQRS model}
\end{figure}
As the capacity of healthcare facilities is limited, during an actual outbreak, the most important task decision-makers have to deal with is to keep the number of infected people to the minimum. This is called flattening the curve. As policies aimed to flatten the curve may only affect parameters \(\langle k \rangle\) and \(\delta\) directly, we investigate their role in relation to the number of people getting infected. We perform sensitivity analysis on a number of possible values for each, leaving all the other parameters fixed.
\begin{figure}[H]
    \centering
    \def\svgwidth{0.7\columnwidth}
    \input{./figures/sensitivity_analysis.pdf_tex}
    \captionsetup{justification=centering,margin=2cm}
    \caption{Sensitivity analysis with regards to \(\langle k \rangle\) and \(\delta\)}
\end{figure}
These figures seem to justify the decisions made by authorities that implemented strict measures during the COVID-19 pandemic. Declaring curfew and promoting self-isolation are some of the most efficient ways to minimize \(\langle k \rangle\). In Wuhan, following the outbreak, authorities were quick to declare curfew which lead to the sharp decline of new cases. Many other countries have also called for social-distancing which is aimed to decrease the interactions between people. More efficient quarantining of individuals who came into contact with the virus also helps immensely. This can be achieved via increasing testing capacities and the quality of testing-kits. Following the COVID-19 outbreak, South Korea was quick to organize large scale testings and has provided drive-through testing facilities to minimize social contacts. These measures resulted in a sharp decline of new COVID-19 cases on the peninsula. On the other hand, countries that failed or were late to put protective measures in place have seen a steep upsurge of newly infected people. The United States Government was particularly slow to act and as a result the country has the most cases of COVID-19 infections. The US President had repeatedly downplayed the significance of the virus and had had advocated for slowing down testings.

\subsection*{Other tasks and conclusion}
Once an epidemic outbreak has been confirmed, having protective measures in place is just one of the many issues public officials face. Another important challenge is to ensure that health-care facilities are provided the appropriate amount of medical supply. Solutions to such problems usually use techniques from operations research. As it turns out, the epidemic diffusion model can help us predict the emergency demand of health-care facilities. A specific such facility has demand at time \(t\) denoted by \(d_t\), and we formulate it as
\begin{gather}
    d_t = \langle k \rangle I(t) + Q(t)
\end{gather}
where \(I\) and \(Q\) refer to the number of infected and quarantined people in the epidemic area. To supply demand, multiple distribution models have been proposed such as the point-to-point distribution model and the multi-depots multiple travelling salesmen model. These two are models considered as pure models, and optimize replenishment efforts according timeliness and cost respectively. In practice, due to their pure nature, they might not be applicable. Other hybrid models  that try to balance between cost and timeliness of delivery have also been proposed. Other arising logistical problems are also connected to the diffusion model.

\bibliographystyle{unsrt}
\begin{thebibliography}{1}
    \bibitem{KZ13}
    E.~Kosygina and M.~P.~W. Zerner
    \newblock Excited random walks: results, methods, open problems
    \newblock In {\em Bulletin of the Institute of Mathematics Academia Sinica (New Series), vol 8, no. 1}, pages. 105--157. 2013

    \bibitem{BW03}
    I.~Benjamini and D.~B.~Wilson
    \newblock Excited random walk
    \newblock In {\em Electronic Communications in Probability, vol 8, no. 9}, pages. 86--92. 2003

    \bibitem{MPRV}
    M.~Menshikov and S.~Popov and A.~Ramirez and M.~Vachkovskaia
    \newblock On a general many-dimensional excited random walk
    \newblock In {\em The Annals of Probability, vol 40, no. 5}, pages. 2106--2130. 2012

    \bibitem{BR07}
    Berard, Jean and Ramirez, Alejandro
    \newblock On a general many-dimensional excited random walk
    \newblock In {\em Electronic Communications in Probability, vol 12, no. 30}, pages. 303--314. 2007

    \bibitem{BS08}
    A.-L. Basdevant and A.~Singh
    \newblock On the speed of a cookie random walk
    \newblock In {\em Probability Theory and Related Fields, vol 141, no. 3-4}, pages. 625--645. 2008


%   \bibitem{basek2013}
%   Gopal Basak and Stanislav Volkov.
%   \newblock Snakes and perturbed random walks
%   \newblock In {\em Proceedings of the Steklov Institute of Mathematics, vol 282, no. 1}, pages. 35--44. Springer, 2013

\end{thebibliography}
\end{document}