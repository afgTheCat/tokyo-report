\documentclass{article}
\usepackage{arxiv}
\usepackage[utf8]{inputenc} % allow utf-8 input
\usepackage[T1]{fontenc}    % use 8-bit T1 fonts
\usepackage{hyperref}       % hyperlinks
\usepackage{url}            % simple URL typesetting
\usepackage{booktabs}       % professional-quality tables
\usepackage{amsfonts}       % blackboard math symbols
\usepackage{nicefrac}       % compact symbols for 1/2, etc.
\usepackage{microtype}      % microtypography
\usepackage{lipsum}
\usepackage{amsthm}
\usepackage{amsmath}
\usepackage{minted}
\usepackage{graphicx}
\usepackage{caption}
\usepackage[thinc]{esdiff}

\newtheorem{theorem}{Theorem}
\newtheorem{lem}[theorem]{Lemma}
\newtheorem{proposition}[theorem]{Proposition}
\theoremstyle{definition}
\newtheorem{definition}[theorem]{Definition}

\title{Research project}

\examineenumber{11070}
\author{Gábor Ádám Fehér}

\begin{document}
\maketitle

\newpage

\section*{Answer to Theme A}
From the fundamental concepts related to mathematical informatics we present the \textbf{Cooley–Tukey algorithm}. The algorithm allows us to compute the discrete Fourier transform of a \(n\) long sequence in \(\mathcal{O}(n\log{}n)\) time. Algorithm with such properties are called fast Fourier transform (FFT) algorithms. Among the many variants of the Cooley–Tukey algorithm, the \textbf{radix-2 DIT} is the simplest and most common one, and thus we present it in great detail. Other forms of the algorithm, as well as other types of FFT algorithms, are also mentioned but no rigorous construction is given.

\subsection*{Mathematical overview}
\begin{definition}[Discrete Fourier transform]
    The discrete Fourier transform (DFT) over the \(n\) dimensional complex field is an invertible linear transformation \(\mathcal{F}: \mathbb{C}^n \to \mathbb{C}^n\). The output of the function for \({(x_k)}_{0 \leq k \leq n - 1}\) is defined by
    \begin{gather}
        X_k = \left(\mathcal{F}(x)\right)_k = \sum_{j=0}^{n-1} x_j e^{-\frac{2{\pi}ikj}{n}} =  \sum_{j=0}^{n-1} x_j \left[ \cos\left( \frac{2{\pi}ikj}{n} \right) - i \sin\left( \frac{2{\pi}ikj}{n} \right) \right],
    \end{gather}
    where the right side of the equation is due to Euler's formula and the trigonometric properties of sine and cosine.
\end{definition}

We should mention that the sign of the exponential is sometimes taken as positive. The resulting equation is equal to the inverse of the previously defined Fourier transformation multiplied by the constant \(n\), that is
\begin{gather}
    x_k = \left(\mathcal{F}^{-1}(X)\right)_k = \frac{1}{n}\sum_{j=0}^{n-1}X_j e^{\frac{2{\pi}ikj}{n}}.
\end{gather}
With some minor adjustments the Cooley–Tukey algorithm is capable of computing the inverse-DFT of the \(n\) long sequence, so no matter which convention we use, the algorithm remains useful.

We introduce some notations. Given \(x = (x_k)_{0 \leq 2n-1}\), we define \(X^{[0]}\) to be the DFT of the even-indexed terms, while \(X^{[1]}\) to be the DFT of the odd-indexed terms in the sequence. That is
\begin{gather}
    X^{[0]}_k = \left(X^{[0]}\right)_k = \left(\mathcal{F}(x_0, x_2, \dots x_{2n-2})\right)_k = \sum_{j=0}^{n-1} x_{(2j)} e^{-\frac{2{\pi}ikj}{n}} = \sum_{j=0}^{n-1} x_{(2j)} e^{-\frac{4{\pi}ikj}{2n}} \label{dq:1} \\
    X^{[1]}_k = \left(X^{[1]}\right)_k = \left(\mathcal{F}(x_1, x_3, \dots x_{2n-1})\right)_k = \sum_{j=0}^{n-1} x_{(2j+1)} e^{-\frac{2{\pi}ikj}{n}} = \sum_{j=0}^{n-1} x_{(2j+1)} e^{-\frac{4{\pi}ikj}{2n}}. \label{dq:2}
\end{gather}
The key observation to make in order to verify the validity of the algorithm is
\begin{gather}
    X_k = \sum_{j=0}^{2n-1} x_j e^{-\frac{2{\pi}ikj}{2n}} = \sum_{j=0}^{n-1} x_{(2j)} e^{-\frac{4{\pi}ikj}{2n}} + e^{-\frac{2{\pi}ik}{2n}}\sum_{j=0}^{n-1} x_{(2j+1)} e^{-\frac{4{\pi}ikj}{2n}}. \label{dq:4}
\end{gather}
Thus for \(0 \leq l \leq n-1\) we have
\begin{gather}
    X_l = X^{[0]}_l + e^{-\frac{2{\pi}ik}{2n}} X^{[1]}_l, \text{ and }
    X_{n+l} = X^{[0]}_l - e^{-\frac{2{\pi}ik}{2n}} X^{[1]}_l, \label{dq:3}
\end{gather}
since
\begin{gather}
    \sum_{j=0}^{n-1} x_{(2j)} e^{-\frac{4{\pi}i(n+l)j}{2n}} = \sum_{j=0}^{n-1} x_{(2j)} e^{-\frac{4{\pi}ilj}{2n}}, \text{ and } e^{-\frac{2{\pi}i(n+l)}{2n}} = -e^{-\frac{2{\pi}il}{2n}}.
\end{gather}

\subsection*{The algorithm}
This basis of all variants of the Cooley–Tukey algorithm is the divide-and-conquer technique. In the radix-2 case the size of the input has to be a power of two. We may add padding zeros to fit the size constraint. From this point on we assume that input vectors are of the right size. Given an \({x}_{0\leq k \leq n}\), where \(2\leq n\), equations \ref{dq:1} and \ref{dq:2} indicate that the input vector should be divided into two parts: the even-indexed and the odd-indexed terms. Once the DFTs of the split vector are calculated, via recursive calls, the DFT of the input vector can be calculated by using equation \ref{dq:3}. If the input vector is one dimensional, its DFT is itself. The following Python code is a working implementation of the radix-2 case, illustrating the key ideas used, but due to its performance it is not meant to be used in real world applications.
\inputminted[xleftmargin=20pt, linenos]{python}{./code/recursive_fft.py}
To evaluate the time complexity of the algorithm, apart from the recursive calls in line 13 and 14, the algorithm runs in \(\Theta(n)\) time, where \(n\) is the length of the input. Using the recurrence relation
\begin{gather}
    T(n) = 2T(n/2) + \Theta(n) = \Theta(n\log{}n).
\end{gather}
With little modification, the Cooley–Tukey algorithm can compute the inverse-DFT of a transformed vector. Similarly to equation \ref{dq:4}, we can derive
\begin{gather}
    x_k = \sum_{j=0}^{2n-1} X_j e^{\frac{2{\pi}ikj}{2n}} = \sum_{j=0}^{n-1} X_{(2j)} e^{\frac{4{\pi}ikj}{2n}} + e^{\frac{2{\pi}ik}{2n}}\sum_{j=0}^{n-1} X_{(2j+1)} e^{\frac{4{\pi}ikj}{2n}}
\end{gather}
so in the 9th line, instead of \(e^{-2{\pi}i/N}\) we have \(e^{2{\pi}i/N}\) for the \(N\) long input, and by multiplying each term of the end result by \(1/n\) given the initial input had size of \(n\), we obtain the original \(x\), within the same time complexity.
\subsection*{Importance and Usage}
The FFT is widely used among different fields of engineering, computer science and mathematics. 

Many popular compression methods, such as \texttt{jpeg} or \texttt{webm}, use discrete cosine transforms (DCT). Algorithms that compute DCTs in \(\mathcal{O}(n\log{}n)\) are known as fast cosine transform (FCT) algorithms. Specialized FCT algorithms exist, that directly compute a vectors DCT, and tough the theoretical running time of such algorithms are better, on modern hardware computing DCTs via FFTs with some \(\mathcal{O}(n)\) pre- and post-processing steps are often faster.

When two degree-bound \(n\) polynomials are represented in the a "extended" point-value form, over \(2n\) distinct points, multiplication can be done in \(\mathcal{O}(n)\) time, by simply multiplying the coefficients. Representing such a polynomial over the \(n\)th root of unity is the same as having the coefficient vectors transformed. Thus multiplying two polynomials can be done in \(\mathcal{O}(n\log{}n)\) time by using FFTs: first transforming the polynomials to their "extended" point value form in \(\mathcal{O}(n\log{}n)\) time, multiplying them in \(\mathcal{O}(n)\), and then transforming back the result to coefficient form. Polynomial representation of the Toeplitz matrices also allow us to multiply them within the same time complexity.

FFts are also widely used in theoretical results. For a long time the \textbf{Schönhage–Strassen algorithm} was the fastest known algorithm for multiplying two large integers. With running time \(\mathcal{O}(n\log{}n\log{}\log{}n)\), it is asymptotically faster than other many traditional multiplication methods, like for example the \textbf{Karatsuba algorithm}, but it only starts to outperform them for numbers with 10000 to 40000 digits, thus it is rarely used in practice. The algorithm employs modular arithmetic instead of the roots of unity (Fourier transforms can be performed in any algebric ring). In 2019 Harvey and van der Hoeven published a \(\mathcal{O}(n\log{}n)\) algorithm, which also builds upon on FFTs. However this is a galactic algorithm, so it only bears theoretical importance.

\section*{Answer to Theme B}
\newpage
\section*{Answer to Theme C}
Mathematical informatics can help immensely to prepare for and to handle infectious diseases. After a brief introduction to the subject, we introduce methodologies to combat each stages of an epidemic. We also take into consideration the recent case of COVID-19, and what are the protective measures that can be implemented and its effects. We should also mention that in the case of an outbreak, the tools of bioinformatics are a useful to analyse the pathogens properties and to develop vaccines but we choose to focus on analytical methods to make precautionary decisions and combat its spread.

\subsection*{Basic concepts}
Compartment models are widely used in modeling infectious diseases. In case of an outbreak, each member of the population is assigned with a label, depending on the individuals relation with the infectious agent. An example would be the labels \textbf{S} for susceptible, \textbf{I} for infectious and \textbf{R} for recovered/removed. Individuals with the same label form a compartment. Individuals may also progress between compartments. The underlying social structure may taken to be homogenous, or can me modeled with small-world networks. Models most use differential equations to simulate the movement of people between different compartments, but stochastic frameworks have also been introduced. 

Generally epidemic containment efforts have four phases: \textbf{preparedness}, \textbf{outbreak investigation}, \textbf{response} and \textbf{evaluation}. These phases contain a wide range of tasks, such as inventory management for essential medical supply and network design for transportation activities during the preparation phase, surveillance system design and implementation during the investigation phase, selection of facilities as PODs (Point of Dispensing) and scheduling of vehicles to be used for transportation during the response phase and identification of bottlenecks during the evaluation phase. Even tough being well prepared is essential, the uncertainties present at an epidemic outbreak dictate the use of real-time solutions. In such an event, decision-makers have to be presented with a wide range of models, that forecast how the epidemic will play out. In the subsequent sections we focus on two key issues they face, the diffusion of the epidemic and the distribution of emergency resources based on demand.

\subsection*{Epidemic modelling}
Different scenarios dictate the use of different models to forecast the diffusion of the infectious agent. Properties of the model depends on the nature of the infectious agent and the policies being implemented. For example if the agent has incubation time, the appropriate compartment needs to be included. The use of quarantine also creates an additional compartment in the model, but this aspect of the model can be controlled by the acting officials. Travel and migration is another aspect which can be controlled. In most countries, the recent COVID-19 outbreak divided people into 5 groups: susceptible people \((S)\), people during incubation period \((E)\), infectious people \((I)\), quarantined people \((Q)\) and recovered people\((R)\). The name of this model is \textbf{SEIQRS model}. We describe a version of this, based on differential equations, and taking the underlying social structure into consideration. The following figure shows the arrangement of the compartments, and how people progress between stages.
\begin{figure}[H]
    \centering
    \def\svgwidth{0.7\columnwidth}
    \input{./figures/model_seiqr.pdf_tex}
    \captionsetup{justification=centering,margin=2cm}
    \caption{The SEIQRS model}
    \label{fig:seiqrs}
\end{figure}
We base the model on small-world networks. The numbers of susceptible people at time \(t\) is \(S(t)\), and the number of people in each compartment is defined analogously. On the above figure, \(\langle k \rangle\) is the average degree distribution of the network, \(\hat{S}\) and \(\hat{I}\) are shorthands for \(S(t-\tau)\) and \(I(t-\tau)\) respectively, where \(\tau\) is incubation period. The rate of change, with which the number of susceptible progresses is
\begin{gather}
    \diff{S}{t} = -\beta\langle k \rangle S(t)I(t) + \gamma R(t).
\end{gather}

All the other differential equations can be derived analogously, by looking at figure \ref{fig:seiqrs}.Depending on given constants, the model shows, that the described systems of differential equations has a disease free stable equilibrium at point \(P = (s, i, q) = (1-e, 0, 0)\) if \(\beta < \frac{d_1 + \delta}{\langle k \rangle (1-e)}\), where \(e\) is a number between 0 and 1, and the lowercase letters of each compartment denotes the proportion of people in the compartment with respect the whole population. Otherwise if \(\beta >  \frac{d_1 + \delta}{\langle k \rangle (1-e)}\),no disease free stable equilibrium exists. This leads us to an important discovery: the stability of the epidemic relies on parameters \(\langle k \rangle, \delta, d_1\), and on the number of exposed people. Of these parameters, active measures can affect \(\langle k \rangle\) and \(\delta\). We will refer to these as key parameters. If the reader wants to be reassured about the validity the above model (particularly about the usage of \(\langle k \rangle\)), and the analytical results, he/she we refer to the results of mean-field theory and Routh-Hurwiz stability criterion. Obtaining the analytical solution for the compartments remains difficult, and thus we have to rely on computation experiments to extract more meaningful data from model. Running \texttt{SciPy}, with parameters \textbf{TODO}, we obtain the following figure. 

During an actual outbreak, the most important task decision-makers have, is to keep the number of infected people to the minimum. With that in mind, leaving all the other parameters fixed, we run sensitivity tests on the key parameters to obtain the following figures.

With regards to the recent COVID-19 pandemic, these figures justify the decisions of authorities that implemented strict measures. Declaring curfew and promoting self isolation are some of the most efficient ways to minimize \(\langle k \rangle\), and as a consequence the number of infected people. In Wuhan, following the outbreak, authorities were quick to declare emergency and declared curfew subsequently, which lead to (compared to other countries) to the sharp decline of new cases. More efficient quarantining of individuals who came into contact with the virus also helps immensely. To achieve such a task, large scale testings have to be organized. Following the outbreak, South Korea was quick to ramp up the production of testing kits and has provided drive-trough testing facilities to maximize the number of test conducted, and minimize social contact. These measures also resulted in a sharp decline of new COVID-19 cases on the peninsula. On the other hand, countries that failed to put in place protective measures, have seen a steep upsurge of infected people. The United States government were slow to act, and as a result have had the most cases of COVID-19 infections. Governmental communication were, in our opinion, harmful. The US president repeatedly downplayed the significance of the virus, and has advocated for slowing down testing.

\subsection*{Emergency resource distribution}
Once an epidemic outbreak has been confirmed, decision-makers face the important issue of ensuring that the appropriate amount of medical supply is provided to health-care facilities. The objective is to periodically replenish such facilities with equipment usually stored in designated stockpiles. As such, the problem relies on techniques from operations research. In this section, we will refer to stockpiles as depots, and facilities in need as demand nodes. The epidemic diffusion models helps us to predict the emergency demand of each demand node during an epidemic. A specific node has demand at time \(t\) with denoted with \(d_t\), and we it formulate as
\begin{gather}
    d_t = \langle k \rangle I(t) + Q(t)
\end{gather}
with respect to the notation used in previous section. To supply demand, multiple distribution methods have been proposed, of which we cover two.
\subsubsection*{}


\bibliographystyle{unsrt}
% \begin{thebibliography}{1}

%   \bibitem{basek2013}
%   Gopal Basak and Stanislav Volkov.
%   \newblock Snakes and perturbed random walks
%   \newblock In {\em Proceedings of the Steklov Institute of Mathematics, vol 282, no. 1}, pages. 35--44. Springer, 2013

% \end{thebibliography}
\end{document}